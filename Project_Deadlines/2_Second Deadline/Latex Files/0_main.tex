\documentclass[12pt]{article}

\usepackage{EngReport}

\graphicspath{{Images/}}
\bibliography{Sources}
\onehalfspacing
\graphicspath{{images/}}
\pagestyle{empty}

\geometry{letterpaper, portrait, includeheadfoot=true, hmargin=0.7in, vmargin=0.5in}

%\fontsize{font size}{vertsize (usually 1.2x)}\selectfont

\begin{document}
\renewcommand{\familydefault}{\rmdefault}
\begin{titlepage}
    \begin{center}
    {\fontsize{35}{48}\selectfont \bfseries Deadline 1} 
    \\\vspace{20pt}
    {\LARGE DBMS Project : \\
    \\\vspace{10pt}
    Defining Project Scope and Requirements} \\
    \vspace{20pt}
    \textbf{{\fontsize{18}{22}\selectfont Shamik Sinha (2022468)}}
        \vspace{8pt}
        \\ {\fontsize{16}{19.2}\selectfont January 22, 2024}
    \end{center}
\end{titlepage}
\pagebreak


\begin{center}
    \fontsize{30}{36}\selectfont\textbf{Conceptual Model}
\end{center}
\fontsize{14}{15}\selectfont{Conceptual models represent high-level abstractions of the data and the relationships between different entities in the system. These models help in designing and understanding the structure of a database without getting into the technical details of how data is stored or accessed.
\\ \\
Here we have showcased the conceptual model using an ER Diagram based on the entities and relationships identified in the first deadline.
\\\vspace{30pt}
\begin{figure}[H]
  \centering
  \includegraphics[width=1\textwidth]{ERDiagram.png}
  \captionof{figure 1. }{Entity Relation Diagram}
  \label{fig:your_label}
\end{figure}

\newpage

\begin{center}
    \fontsize{30}{36}\selectfont\textbf{Relational Model}
\end{center}
\fontsize{14}{15}\selectfont{A Relational Model provides a clear and structured way to represent and manage data. It represents the database as a collection of relations. In other words, they are used to represent how data will be stored in the database.
% \\\vspace{0pt}

\begin{figure}[H]
  \centering
  \includegraphics[width=0.9\textwidth]{RelationalModel.png}
  \\
  \captionof{figure 2. }{Relational Model}
  \label{fig:your_label}
\end{figure}
}
\end{document}
